\documentclass[12pt,a4paper]{article}
\usepackage[utf8]{inputenc}
\usepackage[czech]{babel}
\usepackage[T1]{fontenc}
\usepackage{amsmath}
\usepackage{amsfonts}
\usepackage{amssymb}
\usepackage{graphicx}
\usepackage[left=3cm,right=3cm,top=3cm,bottom=3cm]{geometry}
\author{Ondřej Zelenka}
\begin{document}

\pagenumbering{gobble}

\section*{Závěrečná olympiáda z fyziky, LMFS 2017 - mladší}

\subsection*{1. Centrifuga (12 bodů)}
Fámulové z nejmenovaného matematicko-fyzikálního soustředění chystají let na Měsíc. Kvůli nízkému rozpočtu je jako centrifuga neboli odstředivka použit řetízkový kolotoč o poloměru závěsu $R = 5$ m a délce řetízků $l = 2$ m. Jakou úhlovou rychlostí jej musíme roztočit, abychom dosáhli přetížení 3g?

\subsection*{2. Náklad (10 bodů)}
Ondra Z. s Víťou nakládají vše důležité do rakety. Náklad o hmotnosti $m_1 = 60$ kg mají v bedně, kterou chtějí vytáhnout po nakloněné rovině, která s vodorovnou rovinou svírá úhel $\alpha = 40^\circ$. K tahání používají lano, které je nahoře vedeno přes kladku k volně zavěšenému závaží o hmotnosti $m_2$. Koeficient smykového tření mezi bednou a nakloněnou rovinou je $f = 0.62$. Jaká musí být hmotnost závaží $m_2$, aby se takto bedna s nákladem vytáhla po nakloněné rovině nahoru? Lano i kladka jsou koupené z AliExpressu a tedy nehmotné a bez tření.

%\subsection{Pruser}
%Zatimco probiha nakladani do rakety, na vietnamskeho prodejce nehmotnych lan, jmenem Ti Ti-Sou, se chysta Ceska Obchodni Inspekce, a tedy schovava sve zbozi do zumpy o hustote $\rho_{zumpa} = 1200$ kg$\cdot$m$^{-3}$. Bedna ma tvar krychle o hrane $a = 1$ m a prumernou hustotu $\rho_{bedna} = 900$ kg$\cdot$m$^{-3}$. Jakou praci vynalozi na zatlaceni jedne bedny az na dno, ktere je v hloubce 2 metry? Nebojte, po zatlaceni zbozi na dno ukotvi.

\subsection*{3. Rozpad rakety (12 bodů)}
Naložení věcí úspěšně proběhlo a následuje start. Raketa se úspěšně zvedla ze země a již ve vysoké výšce a v silném vakuu odhazuje své tři motory ve smeru 135$^\circ$ od směru rychlosti rakety, rychlostí o velikosti $v_{motor} = 10$ m$\cdot$s$^{-1}$. Každý motor má hmotnost $m_{motor} = 100$ t a zbytek rakety má hmotnost $m_{zbytek} = 550$ t. O kolik se tak urychlí raketa? Zadané veličiny jsou takto vzhledem k soustavě spojené s raketou před oddělením.

\subsection*{4. Kam vlastně letíme? (8 bodů)}
Ondra Z. se po cestě (ještě nedaleko Země) kouká z okna a snaží se vidět Měsíc, ale nějak ho nevidí. Usoudil, že to je způsobeno lomem světla při průchodu z vakua do ultrastlačeného vzduchu uvnitř rakety. Světlo z Měsíce by dle něj mělo dopadat na okno pod úhlem 50$^\circ$ ke kolmici, index lomu vakua je $n_{vakuum} = 1$ a index lomu ultrastlačeného vzduchu je $n_{vzduch} = 1.3$. O jaký úhel se světlo z Měsíce vychýlí?

\subsection*{5. Oblet (8 bodů)}
Naše výprava se přiblížila k Měsíci a zatímco Ondra B. s lunárním modulem přistává, Víťa zůstal ve velitelském modulu na oběžné dráze kolem Měsíce. Jakou rychlostí musí obíhat, pokud uvažujeme oběh po kruhové trajektorii těsně nad povrchem Měsíce? Měsíc má průměr $d = 3474$ m a hmotnost $M = 7.34\cdot 10^{22}$ kg.

\subsection*{6. Přistání (10 bodů)}
Lunární modul pilotovaný Ondrou B. přistál na Měsíci. Ukázalo se ovšem, že Měsíc není tvořen kameny, ale jedná se o velkou hydrogelovou kouli z AliExpressu. Když se modul začal potápět, Ondra z něj vyskočil a sundal si přilbu, aby urychlil smrt. Jakou barvu vidí Ondra potopený v hydrogelu o indexu lomu $n_{voda} = 1.33$, pokud pozoruje světlo, které mělo ve vakuu modrou barvu o vlnové délce $\lambda = 450$ nm?

\subsection*{7. Přistání vol. 2 (10 bodů)}
Naše výprava se úspěšně vrátila z Měsíce na oběžnou dráhu Země a chce přistát. Při průletu atmosférou ale Ondra Z. nerespektoval nápis "NEOPÍREJTE SE O DVEŘE" a vypadl ven, naštěstí s padákem. Při pádu v atmosféře existuje tzv. terminální rychlost, ke které se padající objekt postupně přibližuje a kterou nikdy nepřesáhne ani při nekonečně dlouhém pádu. Jaká je tato rychlost ve výšce 4478 m n.m.? Ondra je vybaven padákem tvořeným dutou polokoulí o poloměru $r = 5$ m a i s padákem váží $m = 75$ kg, hustota vzduchu je v této výšce $\rho = 750$ g$\cdot$m$^{-3}$; koeficient aerodynamického odporu je $C = 1.3$.

\subsection*{8. Matterhorn (8 bodů)}
Jak lze z uvedené nadmořské výšky v minulé úloze uhodnout, Ondra kvůli neschopnosti navigovat padák přistál přímo na vrcholu Matterhornu. Naštěstí i na tuto variantu je připraven a slaňuje dolů. Na jakou teplotu se zahřeje slaňovací pomůcka, pokud na vrcholu měla $0^\circ$C (je tam kosa), slaňuje dolů k Hörnlihütte do výšky 3260 m n.m., jistící pomůcka má tepelnou kapacitu $C = 10$ kJ$\cdot$K$^{-1}$ a veškerá energie získaná slaňováním se promění na teplo? Vybavení není z AliExpressu a tedy má tření; po odhození padáku Ondra váží jen 70 kg.

\subsection*{9. Oslava (12 bodů)}
Na oslavu úspěchu narazil Filip nový sud a natočil PerMovi pivo do válcového půllitru. Jelikož raketu zaplatil penězi na stipendia, utíká nyní před Národní centrálou proti organizovanému zločinu. Před Kačákem našel odložený skateboard bez tření (AliExpress) a sjíždí na něm dolů k úpravně vody, dno půllitru je rovnoběžné s plochou skateboardu. Jaký tvar má hladina v půllitru?

\subsection*{10. Velké finále (10 bodů)}
Přistávací modul s Víťou přistál v moři a kvůli chybě ve výpočtech se potopil do hloubky 15 m, než se jej podařilo zastavit. Nyní jej hledá pobřežní hlídka. Při jeho hledání ale nebere v potaz totální odraz světla při průchodu z vody do vzduchu. Jak blízko musí být k Víťovi a jeho modulu, aby jej mohla vidět při pohledu do vody? Jako správná odpověď se počítá jak délka spojnice modulu a pobřežní hlídky, tak délka jejího průmětu do roviny moře.

\end{document}