\documentclass[12pt,a4paper]{article}
\usepackage[utf8]{inputenc}
\usepackage[czech]{babel}
\usepackage[T1]{fontenc}
\usepackage{amsmath}
\usepackage{amsfonts}
\usepackage{amssymb}
\usepackage{graphicx}
\usepackage[left=3cm,right=3cm,top=3cm,bottom=3cm]{geometry}
\usepackage{enumitem}

\begin{document}

\begin{center}
    \textbf{\Large Průběžná olympiáda z fyziky starších}
    \vspace{1em}
    
    \textbf{Odevzdání do 22:59:59 SEČ, 4. 8. 2023 gregoriánského kalendáře}
\end{center}
\vspace{1em}

\begin{enumerate}[label=\arabic*)]

\item (3 body) Mějme vektorové pole
\begin{equation*}
    \vec{A}\left(x,\, y,\, z\right) = \left(ye^{-\left(x^2+y^2\right)},\, -xe^{-\left(x^2+y^2\right)},\, 0\right) ~.
\end{equation*}
Spočtěte jeho rotaci a divergenci.

\item (3 body)
Vyjádřete divergenci rotace vektorového pole $\vec{B}$ a zjednodušte ji.

\item (2 body) Proton a elektron v atomu vodíku se navzájem přitahují jak elektrostatickou, tak gravitační silou. Jaký je podíl velikostí těchto sil mezi nimi? Jako jejich vzdálenost uvažujte Bohrův poloměr $a_0 \doteq 0.529\cdot 10^{-10}\,\mathrm{m}$. Jak se odpověď změní při jiných hodnotách vzdálenosti?

\item (4 body) V každém vrcholu čtverce o straně $a$ je umístěn bodový náboj $q$. Určete intenzitu a potenciál elektrického pole ve středu čtverce. Jaký náboj $Q$ je třeba umístit doprostřed čtverce, aby síly působící na ostatní náboje byly nulové?

\item (5 bodů) Představte si sféru (kulové slupky) o poloměru $R$, nabitou rovnoměrně nábojem $Q$. Spočtěte intenzitu elektrického pole $\vec{E}\left(\vec{r}\right)$ a potenciál $\varphi\left(\vec{r}\right)$ pro každé $\vec{r}\in\mathbb{R}^3$.

\item (3 body) Spočtěte následující integrály:
\begin{align*}
    &\int \frac{2x-7}{3-x}\,\mathrm{d}x ~,\\
    &\int_0^\pi \cos^n\left(x\right)\sin\left(x\right)\mathrm{d}x ~,\quad n\in\mathbb{N} ~,\\
    &\int_{-\pi}^{\pi} x^4\sin^3\left(x\right)\mathrm{d}x ~.
\end{align*}

\end{enumerate}

\iffalse

\newpage
\begin{center}
    \textbf{\Large Průběžná olympiáda z fyziky starších}
    \vspace{1em}
    
    \textbf{Odevzdání do 22:59:59, 6. 8. 2023 gregoriánského kalendáře}
\end{center}

\begin{enumerate}[resume,label=\arabic*)]

\item (5 bodů) Další úloha.

\end{enumerate}

\fi

\end{document}
