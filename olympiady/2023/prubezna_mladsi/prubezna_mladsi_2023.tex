\documentclass[12pt,a4paper]{article}
\usepackage[utf8]{inputenc}
\usepackage[czech]{babel}
\usepackage[T1]{fontenc}
\usepackage{amsmath}
\usepackage{amsfonts}
\usepackage{amssymb}
\usepackage{graphicx}
\usepackage[left=3cm,right=3cm,top=3cm,bottom=3cm]{geometry}
\usepackage{enumitem}

\begin{document}

\begin{center}
    \textbf{\Large Průběžná olympiáda z fyziky mladších}
    \vspace{1em}
    
    \textbf{Odevzdání do 22:59:59 SEČ, 3. 8. 2023 gregoriánského kalendáře}
\end{center}
\vspace{1em}

\begin{enumerate}[label=\arabic*)]

\item (3 body) Mějme vektory
\begin{equation*}
    \vec{u} = \left(4,\, 3.5,\, \frac{2}{3}\right),\quad \vec{v} = \left(0,\, 3,\, -4\right) ~.
\end{equation*}
Spočtěte $\vec{w} = \vec{u}\times\vec{v}$ a pomocí skalárního součinu určete úhly, které mezi sebou jednotlivé vektory $\vec{u}$, $\vec{v}$ a $\vec{w}$ svírají.

\item (4 bodů)
Mějme vektory $\vec{a}$, $\vec{b}$ a $\vec{c}$. Vyjádřete dvojitý vektorový součin $\vec{a}\times\left(\vec{b}\times\vec{c}\right)$ pomocí skalárních součinů. Hint: rozepište si celý výraz do složek :-)

\item (3 body)
Víťa si zlomil nohu v bérci. Z jeho kolena do zlomeniny vede vektor $\vec{a} = \left(1,\, -1,\, -5\right)$ a ze zlomeniny ke kotníku $\vec{u} = \left(-1,\, 1,\, -1\right)$. Pomocí vektorů $\vec{a},\vec{u}$ spočtěte, o jaký úhel je třeba narovnat jeho nohu, aby zase mohl chodit, než si zlomí druhou.

%\item ()
%Mějme vektor $\vec{u} = \left(1,\, 1,\, -1\right)$. Určete vektor $\vec{v}$ v rovině $xy$, aby vektor $\vec{u}\times\vec{v}$ mířil svisle (podél osy $z$) a svou délkou akorát dosáhl na Měsíc.

\item (4 body) Spočtěte:
\begin{align*}
    &\frac{\mathrm{d}^n}{\mathrm{d}x^n}x^n,\quad n\in\mathbb{N} ~,\\
    &\frac{\mathrm{d}}{\mathrm{d}x}\left[\frac{1}{\mathrm{ln}\left(x\right)}\right] ~,\\
    &\frac{\mathrm{d}}{\mathrm{d}x} e^{\cos\left(x^{\sin\left(\pi\right)}\right)}     ~,\\
    &\frac{\mathrm{d}}{\mathrm{d}x}\left[x^3\sin\left(x\right)\right] ~.
\end{align*}

%\item (3 body) Spočtěte následující integrály:
%\begin{align*}
%    &\int \sin\left(5x\right)\mathrm{d}x ~,\\
%    &\int \sin\left(x\right)\cos\left(x\right)\mathrm{d}x ~,\\
%    &\int_{-13}^{13} x^4\sin\left(x\right)\mathrm{d}x ~.
%\end{align*}

\end{enumerate}

\iffalse
\newpage
\begin{center}
    \textbf{\Large Průběžná olympiáda z fyziky mladších}
    \vspace{1em}
    
    \textbf{Odevzdání do 22:59:59, 6. 8. 2023 gregoriánského kalendáře}
\end{center}

\begin{enumerate}[resume,label=\arabic*)]

\item (3 body) Určete limity:
\begin{align*}
    &\lim_{x\to 1} \frac{\sin\left(x\right)}{x} ~,\\
    &\lim_{x\to 1} \frac{x^3-1}{x-1} ~.\\
\end{align*}

\item (6 bodů) Ještě nějaká úloha.

\end{enumerate}
\fi

\end{document}
