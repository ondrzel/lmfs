\documentclass[10pt,a4paper,landscape]{article}
\usepackage[utf8]{inputenc}
\usepackage[czech]{babel}
\usepackage[T1]{fontenc}
\usepackage{amsmath}
\usepackage{amsfonts}
\usepackage{amssymb}
\usepackage{graphicx}
\usepackage{multicol}
\usepackage[left=1cm,right=1cm,top=0.2cm,bottom=1cm]{geometry}
\usepackage{enumerate}

\author{Ondřej Zelenka}
\setlength{\columnsep}{1cm}
\begin{document}
\pagestyle{empty}
\textbf{\center\LARGE Závěrečná paralympiáda mladších, LMFS 2023: Nanomagnetická odysea}

\begin{multicols}{2}[]

\section{Nanoprolog (10 nanobodů)}
Nejmenovaná večerní přednášející přijíždí na LMFS a přiváží s sebou experiment s nanomagnety. Zavěsila jej na nanovlákno o délce $l$ v nákladním prostoru nanokamionu a rozjíždí se po rovině s rovnoměrným zrychlením o velikosti $a_K$ dopředu. Jaká bude rovnovážná poloha nanokyvadla tvořeného zavěšeným experimentem?

\section{Kvantový delfín (8 nanobodů)}
Víťa snědl poslední zbytky jídla na chatě a stále má hlad, tak se rozhodl ulovit si delfína, který do Slap kvantově protuneloval z oceánu. Z Cholínského mostu spustil obruč se sítí, která pak visí 5 metrů nad hladinou, a $110\,\mathrm{kg}$ těžký delfín vyskočil přímo do sítě! Jak rychle se musel vymrštit z hladiny, aby vyskočil až k obruči?

\section{Mokré G (14 nanobodů)}
Tomáš roznáší Bloudirint a rozhodl se písmeno G schovat do Slap. Sroloval papír a uzavřel jej do válcové nádoby o výšce $230\,\mathrm{mm}$ a průměru $100\,\mathrm{mm}$. Zapomněl ji ovšem zatížit, takže po vložení do přehrady kvůli vztlakové síle plave ponořená pouze do čtvrtiny své výšky. Kolik vynaloží práce na její ponoření do hloubky $30\,\mathrm{cm}$ (myšleno hloubka horní podstavy)?

\section{Táhni (12 nanobodů)}
Na celodenním výletu, po vytažení potopeného mola z vody, pomohl Tomáš navrhnout nový systém upevnění pomocí kubických pružin, které v podélném směru působí silou $F_x = -Kx^3$. Jaký tvar má jejich potenciální energie?

\section{Aby jízda měla jiskru a švih! (10 nanobodů)}
Vojta se rozhodl překonat rekordní vzdálenost při konstrukční hře, sestavil skateboard, rozjel se po paletě a bez tření sjíždí z kopce se sklonem $\alpha$. Načepoval si k tomu kelímek piva. Jaký tvar má během sjezdu tvar hladiny v kelímku?

\section{Proč si policisté berou do auta pilu? (8 nanobodů)}
Ve spodní části sjezdu se cesta srovnala a Vojta najednou musí zatáčet, aby nenaboural do některého zaparkovaného auta. Koeficient smykového tření mezi kolečky a silnicí je $f = 0.1$ a poloměr zatáčky je 5 metrů. Jak rychle může jet, aby nenaboural? Zodpovězte otázku v názvu úlohy. [1 nanobod]
\columnbreak

\section{Řekl někdo kočička? (10 nanobodů)}
Tomáš v noci se skupinou účastníků sbírá Bloudirint. Po nálezu posledního písmene ovšem zabloudili a došli na kosmodrom Bajkonur na souřadnicích $\mathrm{N}45.965^\circ$, $\mathrm{E}063.305^\circ$. Domluvili si, že je zpět na Hořovickou chatu (na souřadnicích $\mathrm{N}49.720^\circ$, $\mathrm{E}014.331^\circ$) donese raketa s doletem 4000 km. Bude to stačit? Pro převod geografických souřadnic $\theta$ a $\phi$ (zeměpisná šířka, délka) na kartézské můžete použít například vztahy
\begin{subequations}
\begin{align}
x &= r\cos\theta\cos\phi ~,\\
y &= r\cos\theta\sin\phi ~,\\
z &= r\sin\theta ~.
\end{align}
\end{subequations}

\section{Suché G (12 nanobodů)}
Účastníci nastoupili do rakety a vrátili se na chatu. Tomáš ovšem nastoupil do špatné a byl vystřelen do vesmíru. Jakou nejmenší rychlost musí na povrchu Země mít, aby jen setrvačností doletěl neomezeně daleko? Neuvažujte odpor atmosféry. Potenciální energie gravitační síly mezi hmotnými body o hmotnostech $m_1$, $m_2$ je
\begin{equation}
E_p\left(\vec{r}\right) = -G\frac{m_1m_2}{\left|\vec{r}\right|} ~.
\end{equation}

\section{Shitsucker (8 nanobodů)}
Fekální vůz vysál jímku pod chatou a z Cholínského mostu, vysokého 11 metrů, vylévá svůj náklad do přehrady $20\,\mathrm{cm}$ širokým otvorem ve své spodní části. Proud má počáteční rychlost $5\,\mathrm{m}\cdot\mathrm{s}^{-1}$. Jaký je profil proudu a jakou rychlostí dopadá na hladinu?

\section{Sihnus a cosihnus (8 nanobodů)}
Spočtěte následující integrály:
\begin{subequations}
\begin{align}
&\int_0^\pi \sin^5\left(x\right)\,\mathrm{d}x ~,\\
&\int_0^{2\pi} \cos^3\left(x\right)\cdot\cos\left(\Gamma\left(10^{12}\right)\right)\,\mathrm{d}x ~,
\end{align}
\end{subequations}
kde gama funkce je definována jako $\Gamma\left(z\right) = \int_0^\infty y^{z-1}e^{-y}\,\mathrm{d}y$.

\section*{(Ne)užitečné konstanty:}
tíhové zrychlení $g = 9.80665\,\mathrm{m}\cdot\mathrm{s}^{-1}$, hustota vody $\rho_{\mathrm{H}_2\mathrm{O}} = 997\,\mathrm{kg}\cdot\mathrm{m}^{-3}$\\
hustota obsahu fekálního vozu $\rho_\mathrm{H} = 1020\,\mathrm{kg}\cdot\mathrm{m}^{-3}$, poloměr Země $R_\mathrm{Z} = 6378\,\mathrm{km}$\\
gravitační konstanta $G = 6.67\cdot 10^{-11}\, \mathrm{m}^3\cdot\mathrm{kg}\cdot \mathrm{s}^2$, Vojtova hmotnost $\rho_V = 69\,\mathrm{kg}$

\end{multicols}

\end{document}
