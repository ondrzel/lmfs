\documentclass[10pt,a4paper,landscape]{article}
\usepackage[utf8]{inputenc}
\usepackage[czech]{babel}
\usepackage[T1]{fontenc}
\usepackage{amsmath}
\usepackage{amsfonts}
\usepackage{amssymb}
\usepackage{graphicx}
\usepackage{multicol}
\usepackage[left=1cm,right=1cm,top=0.2cm,bottom=1cm]{geometry}
\usepackage{enumerate}

\author{Tomáš Dolák}
\setlength{\columnsep}{1cm}
\begin{document}
\pagestyle{empty}
	\textbf{\center\LARGE Závěrečná olympiáda z fyziky mladších, LMFS 2024: Nanoslečna drsňák II, ještě drsnější}
\vfill

\begin{multicols}{2}[]

\section{Vojta rudne (8 nanobodů)}
Vojta se rozčiluje, že nelze ubránit Filipa na frisbee. Filip s nataženou rukou a povyskočením dosahuje výšky $3~\mathrm{m}$. Vojta měří k ramennímu kloubu $1.6~\mathrm{m}$ a má paži dlouhou $80~\mathrm{cm}$. Napište nejmenší rychlost, kterou musí Vojta vyskočit z pokrčené pozice, kdy své tělo sníží o $20~\%$, aby trefil frisbee držené Filipem.

\section{Skóruje Filip? (12 nanobodů)}
Filip hází své slavné dlouhé hody na flašku frisbeem. Mírně ale přehodil -- frisbee vyletělo z atmosféry, po elipse jednou oběhlo Slunce a srazilo flašku během hry na LMFS 2026. Ve svém periheliu (nejbližší bod trajektorie ke Slunci) jen tak tak minulo Merkur ve vzdálenosti $0.387~\mathrm{au}$ od Slunce. Spočtěte excentricitu trajektorie frisbee a afélium (nejvzdálenější bod od Slunce). Prolétne pásem asteroidů, který je 2 až 4 au od Slunce?


%Jako na potvoru zrovna fouká vítr rychlostí $\vec{v}$ rovnoběžně se zemí a kolmo na osu flašek. Filip stojící uprostřed hřiště, přesně mezi flaškami chce trefit flašku soupeře vzdálenou $12~\mathrm{m}$. Určete vektor síly, kterou musí Filip vynaložit na frisbee o hmotnosti $m_\text{frisbee}$, aby skóroval.

\section{Mokré G (12 nanobodů)}
Tomáš roznáší Bloudirint a rozhodl se písmeno G schovat do Slap. Sroloval papír a uzavřel jej do válcové nádoby o výšce $230\,\mathrm{mm}$ a průměru $100\,\mathrm{mm}$. Zapomněl ji ovšem zatížit, takže po vložení do přehrady kvůli vztlakové síle plave ponořená pouze do čtvrtiny své výšky. Kolik vynaloží práce na její ponoření do hloubky $30\,\mathrm{cm}$ (myšleno hloubka horní podstavy)?

\section{Účastnice rudnou (8 nanobodů)}
Tomáš prochází kolem rudnoucích účastnic, které pohodlně sedí v jednonohém wallsitu. Jakou silou drží stehna účastnic zbytek jejich těl, pokud délka stehna je $l_\text{stehna}$ a hmotnost účastnice držená stehny je $m_\text{účastnice}$, přičemž správný wallsit znamená pravý úhel mezi lýtkem a stehnem? Jakou práci účastnice tímto cvikem vykonají? Hlavně popište své myšlenky, jak by to vůbec šlo provést!

\section{Laxní mravenec (12 nanobodů)}
Mravenec leze po vteřinové ručičce analogových hodin. V počátku všeho byl na okraji $20~\mathrm{cm}$ dlouhé vteřinové ručičky a vyrazil pravidelným tempem $1~\mathrm{cm}$/$5~\mathrm{s}$ postupovat do středu hodin. Ve středu hodin bez zbytečného zdržování přeskočil na minutovou ručičku, a neboť bylo právě pravé poledne, vyrazil vzhůru po minutovce laxnějším tempem $1~\mathrm{cm}$/$60~\mathrm{s}$, dokud nedošel na konec $25~\mathrm{cm}$ dlouhé minutové ručičky. Urči vektory polohy, rychlosti a zrychlení mravence v čase.
\columnbreak

\section{Helča \& Tom (8 nanobodů)}
Tomáš tvořený ze 100\,\% Kofolou ($=\mathrm{H}_2\mathrm{O}$) ztratil z $1\,\%$ svých atomů po jednom elektronu a Helča mu je poctivě sesbírala. Jak velkou silou jsou k sobě na vzdálenost $d$ přitahováni? Hmotnost Toma je BÚNO $90~\mathrm{kg}$. 


\section{Kdo vycvičí Filipra? (7 nanobodů)}
Cvičený delfín Filiper plave ohromnou rychlostí $60~\mathrm{km}\cdot\mathrm{h}^{-1}$. Své 34 let staré $100~\mathrm{kg}$ těžké tělo pak beze ztrát energie vymrští nad hladinu. Zvládne trefit racka letícího $28~\mathrm{m}$ vysoko nad hladinou?

\section{Alex \& Víťa (10 nanobodů)}
Víťa vidí Alex, jak k němu běží rychlostí $15~\mathrm{km}\cdot\mathrm{h}^{-1}$. Jelikož oba už byli pod vlivem, nestrefili se přímo, ale Alex narazila do Víťovy paže a systém těchto dvou těl se začal otáčet. Jakou rychlostí? Alex má hmotnost $2/5$ Víťovy hmotnosti a Víťova natažená paže měří $83~\mathrm{cm}$.

\section{Perm permuje (9 nanobodů)}
	Perm se zděsil, že došlo pivo. Naštěstí našel pod objektem zapomenutý sud piva o hmotnosti $69~\mathrm{kg}$ a rozhodl se ho dopravit do sklepa objektu. Naložil si sud na sáňky a táhl je za sebou po vzdálenost $100~\mathrm{m}$ do kopce s mírným sklonem vychlazené plzně $12^{\circ}$, brzděn třením s koeficientem $0.3$. Když se jedná o pivo, má schopnost vyvinout sílu $400~\mathrm{N}$. Dotáhne sud do sklepa? Spočtěte Permem vykonanou práci.

\section{Perm rudne, Víťa zrudne (14 nanobodů)}
Víťa má pytel a ukazuje ho účastnicím. Perm zuří a tak mu do něj vystřelí ze vzdálenosti $6~\mathrm{m}$ od Víti kulku o hmotnosti $8~\mathrm{g}$ s počáteční rychlostí $750~\mathrm{m}\cdot\mathrm{s}^{-1}$. Pytel má rozměr v trajektorii střely $1~\mathrm{m}$. Hmota uvnitř pytle působí odporovou silou úměrnou rychlosti kulky s koeficientem úměrnosti $0.1~\mathrm{kg}\cdot\mathrm{s}^{-1}$. Na proražení látky na druhém konci pytle je třeba hybnost kulky $1.6~\mathrm{kg}\cdot\mathrm{m}\cdot\mathrm{s}^{-1}$ (při vstupu hybnost neztrácí). Prostřelí Perm Víťovi pytel?

\section*{(Ne)užitečné konstanty:}
tíhové zrychlení $g = 9.80665\,\mathrm{m}\cdot\mathrm{s}^{-1}$, hustota vody $\rho_{\mathrm{H}_2\mathrm{O}} = 997\,\mathrm{kg}\cdot\mathrm{m}^{-3}$\\
hustota obsahu fekálního vozu $\rho_{\mathrm{H}} = 1020\,\mathrm{kg}\cdot\mathrm{m}^{-3}$\\
střední vzdálenost Země od Slunce $1~\mathrm{au} \doteq 1.5\cdot 10^{20}~\mathrm{nm}$

\end{multicols}

\end{document}

