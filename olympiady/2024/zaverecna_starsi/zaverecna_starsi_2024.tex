\documentclass[11pt,a4paper,landscape]{article}
\usepackage[utf8]{inputenc}
\usepackage[czech]{babel}
\usepackage[T1]{fontenc}
\usepackage{amsmath}
\usepackage{amsfonts}
\usepackage{amssymb}
\usepackage{graphicx}
\usepackage{multicol}
\usepackage[left=1cm,right=1cm,top=0.2cm,bottom=1cm]{geometry}
%\usepackage{enumerate}

\author{Tomáš Dolák}
\setlength{\columnsep}{1cm}
\begin{document}
\pagestyle{empty}
\textbf{\center\LARGE Závěrečná olympiáda z fyziky starších, LMFS 2024: Good luck fellas.}
\vfill

\begin{multicols}{2}[]

\section{Bubble Trouble (7 luckbodů)}
Bublinka v krychlovém krystalu o délce strany $a$ indexu lomu $n=\sqrt{3}$ se při pohledu pod úhlem $60^\circ$ od kolmice jeví býti v hloubce $a\cdot\sqrt{3}/2$. Kde je ve skutečnosti? Good luck fellas.

\section{Polární expres (7 luckbodů)}
Sestavte soustavu lineárních polarizátorů tak, aby výsledná intenzita prošlého světla byla $1/6$ nepolarizovaného světla a přitom bylo světlo otočeno o pravý úhel vzhledem k prvnímu polarizátoru. Good luck fellas.

\section{BaRé jiskra (12 luckbodů)}
Máme elektrické pole 
%\begin{equation}
%    \vec{E}=\left(0, E_{0y}\mathrm{cos}\left(kz-\omega t - \pi\right),E_{0z}\mathrm{sin}\left(kz + ky -\omega t - \pi\right)\right),
%\end{equation}
\begin{equation}
\vec{E} = \left(E_{0x}\cdot\cos\left(\frac{k}{\sqrt{2}}\cdot\left(y+z\right) - \omega t - \pi\right),0,0\right)
\end{equation}
a vaším úkolem je vyjádřit magnetické pole, které kolem tohoto elektrického vzniká. Good luck fellas.

\section{Minimální odchylka od normálu (10 luckbodů)}
Bílé světlo prochází hranolem s průřezem rovnoramenného trojúhelníku a úhlem mezi rameni $\varphi$. Pod jakým úhlem musí vstupovat světlo do hranolu, aby celková odchylka (tj. odchylka po prvním plus odchylka po druhém průchodu paprsku) byla co nejmenší? Vyjádřete i tuto odchylku. Good luck fellas.

\section{Měli jste dávat pozor... (10 luckbodů)}
Odvodťe Snellův zákon z Fermatova principu nejkratšího času. Good luck fellas.

\section{To je válec! (10 luckbodů)}
Vytvořte předpis pro válcově symetrické vektorové pole, které splňuje vlnovou rovnici. Good luck fellas.

\section{Nonabsorbum (12 luckbodů)}

Neabsorbujícím prostředím s indexem lomu $n$ se šíří kruhově polarizovaná vlna popsaná rovnicemi
\begin{subequations}
\begin{align}
E_x &= E_0\cos{\left(kz - \omega t\right)},\\
E_y &= E_0\sin{\left(kz - \omega t\right)}.
\end{align}
\end{subequations}

Vypočtěte objemovou hustotu její elektrické a magnetické energie a výkon na jednotku plochy. Good luck fellas.

\section{Narovnejte Tomášovi kládu (8 luckbodů)}
Tomovi se podařilo zlomit tyč do úhlu $30^{\circ}$. Rozhodne se, že ji ponoří do kapaliny, aby se díky lomu světla jevila rovná. Navrhněte vhodný index lomu a úhly ponoření tyče a pohledu, aby se mu to povedlo. Good luck fellas.

\section{Nanodetektor (10 luckbodů)}
 Na detektor o ploše $10^{14}~\mathrm{nm^2}$ dopadá rovinná vlna červeného kruhově polarizovaného světla. Detektor změřil výkon $10^6~\mathrm{nW}$. Vypočtěte velikost amplitudy elektrické a magnetické složky vlny. Good luck fellas.

\section{Good luck fellas. (14 luckbodů)}

Odvoďte Jonesovu matici pro rotátor, tj. odvoďte matici, která popíše průchod světla rotátorem (prvek, který stáčí lineárně polarizované světlo o úhel $\varphi$).

\section*{(Ne)užitečné konstanty}
gravitační konstanta $G = 6.67\cdot 10^{-11}\, \mathrm{m}^3\cdot\mathrm{kg}\cdot \mathrm{s}^2$\\
hustota obsahu fekálního vozu $\rho_{\mathrm{H}} = 1020\,\mathrm{kg}\cdot\mathrm{m}^{-3}$\\
konstanta jemné struktury $\alpha = 1/137$

\end{multicols}

\end{document}

