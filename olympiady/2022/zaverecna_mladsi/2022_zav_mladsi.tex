\documentclass[10pt,a4paper,landscape]{article}
\usepackage[utf8]{inputenc}
\usepackage[czech]{babel}
\usepackage[T1]{fontenc}
\usepackage{amsmath}
\usepackage{amsfonts}
\usepackage{amssymb}
\usepackage{graphicx}
\usepackage{multicol}
\usepackage[left=1cm,right=1cm,top=0.5cm,bottom=1cm]{geometry}
\usepackage{enumerate}

\author{Ondřej Zelenka}
\begin{document}
\pagestyle{empty}
\textbf{\center\LARGE Závěrečná paralympiáda mladších, LMFS 2022: Ekvádorské elegie}

\begin{multicols}{2}[]

\section{Gravíťační prak (10 bodů)}
Víťa byl personálem ÚTF deportován z ČR, protože kázal, že $\pi=3$, a $\pi^2=10$. Nejmenovaný profesor napne Víťu o hmotnosti 92 kg na praku popsaném silou $F = -kx$ s tuhostí $k = 2\cdot 10^5 ~\mathrm{kg}\cdot\mathrm{s}^{-2}$ na délku $l$ a vystřelí jej po úhlem $45^\circ$. Chce Víťu dostřelit na saskou stranu Krušnohoří 90 kilometrů daleko, o 500 metrů výše. Jaká musí být délka $l$?

%\section{Sjezd (10 bodů)}
%Víťa si načepoval pivo do válcového kelímku a na skateboardu, který měl v krabici s experimenty, sjíždí z Krušných hor. Silnici tvoří nakloněná rovina pod úhlem $\alpha$, ložiska skateboardu jsou dokonale promazaná a Víťa se bez odporu vzduchu rozjíždí z kopce. Jaký tvar má hladina piva?

\section{Proč si policisté berou do auta pilu? (8 bodů)}
Víťa sjíždí na skateboardu, který měl v krabici s experimenty, z Krušných hor. Po sjezdu z kopce se silnice srovnala, ale přišla ostrá neklopená zatáčka! Koeficient smykového tření mezi kolečky skateboardu a povrchem silnice je $f=0.1$ a poloměr zatáčky je 10 metrů. Jak rychle může Víťa jet, aby zatáčku vybral? Zodpovězte otázku v názvu úlohy [1 bod].

\section{Kamiont (8 bodů)}\label{sec:kamion}
Víťa si stopnul kamion a cestuje do nizozemského přístavu. V návěsu zavěsil kyvadlo délky~$l$ během rozjíždění s konstantním zrychlením kamionu $a_K$. Jaká je rovnovážná poloha kyvadla?

\section{Adolphin (8 bodů)}
Ostatní fámulové se s Víťou setkali v přístavu a pokračují společně. Na lodi v Atlantiku dostali hlad a rozhodli se ulovit si delfína. Drží síť ve výšce 7 metrů nad hladinou a delfín vyskočil z vody přímo do sítě! Jakou musel plavat rychlostí, aby doskočil tak vysoko?

\section{Příprava do starších (8 bodů)}
Ondra se mezitím připravuje na cvičení z kvantové fyziky u starších a počítá s atomem vodíku. Jakou rychlostí v něm obíhá elektron kolem protonu? Na kruhové trajektorii jej udržuje Coulombova síla
\begin{equation}
\vec{F}_C = \frac{1}{4\pi\epsilon_0}\frac{q_p q_e}{\left|\vec{r}\right|^2}\cdot\frac{\vec{r}}{\left|\vec{r}\right|} \,.
\end{equation}
Hmotnosti protonu a elektronu jsou $m_p = 1.67\cdot 10^{-27}~\mathrm{kg}$, $m_e = 9.11\cdot 10^{-31}~\mathrm{kg}$, náboje jsou $q_p = -q_e = 1.602\cdot 10^{-19}~\mathrm{C}$, permitivita vakua je $\epsilon_0 = 8.85\cdot 10^{-12}~\mathrm{kg}^{-1}\cdot\mathrm{m}^{-3}\cdot\mathrm{s}^{2}\cdot\mathrm{C}^{2}$, a elektron obíhá proton ve vzdálenosti $5.29\cdot 10^{-11}~\mathrm{m}$.

\section{Sladká pomsta s.r.o. (14 bodů)}
Víťa se rozhodl pomstít ÚTF za své vyhnanství zničením budovy MFF v Tróji. Hacknul systémy NASA a změnil dráhu ISS, která obíhala 350 km nad zemským povrchem. Krátkým zážehem motorů přímo proti směru pohybu ji zpomalil bez okamžité změny směru rychlosti a její nové přízemí je v přízemí trojské budovy. O kolik změnil Víťa její rychlost?

\section{Ekvádor (10 bodů)}
Fámulové dorazili do exilu v Ekvádoru a na rovníku sledují show pro turisty, jak se voda v záchodě točí na různou stranu na různých polokoulích. Jako konkurenci sestaví pořádný experiment, ve kterém je tento jev skutečně vidět jako důsledek Coriolisovy síly. Na jakou stranu se bude vír stáčet na které polokouli, a proč?
%Na jakou stranu na které, a proč? Rozmyslete si důkladně směry a uvažujte, že je Coriolisova síla dostatečně silná, že způsobí znatelný efekt.

\section{Ekvádor Reloaded (12 bodů)}
Tomáš vysvětluje teoretický popis experimentu bulharským turistům tak nadšeně, až poskakuje po trajektorii popsané polohovým vektorem ($A$ a $\alpha$ jsou konstanty)
\begin{equation}
\vec{r}\left(t\right) = \left(2A\cos\left(\alpha t\right), A\sin\left(2\alpha t\right), 0\right) \,.
\end{equation}
\begin{enumerate}
\item Popište Tomášovu trajektorii: jaký má tvar, je periodická (popř. s jakou periodou)? Nejlépe ji nakreslete! (nebo alespoň Tomáše s tabulí [2 body])
\item Spočtěte rychlost a zrychlení.
\end{enumerate}

\section{Ekvádor Levorutions (10 bodů)}
Díky poctivému teoretickému popisu experimentu bylo fámulům povoleno vrátit se do ČR, dopluli zpět do Evropy a vrací se kamionem z úlohy \ref{sec:kamion} do ČR. Kamion opět zrychloval stejně, a pak přestal zrychlovat při rychlosti $90~\mathrm{km}\cdot\mathrm{h}^{-1}$, načež se kyvadlo rozkývalo a jeho úhlová výchylka z rovnovážné polohy je přibližně $\phi\left(t\right) = \Phi\cos\left(\alpha t\right)$,
%\begin{equation}
%\phi\left(t\right) = \Phi\cos\left(\alpha t\right) \,,
%\end{equation}
kde $\Phi$ je amplituda z úlohy \ref{sec:kamion} a $\alpha$ je konstanta. Tomáš sleduje mravence lezoucího z konce kyvadla na vrchol rychlostí $v_m$ vzhledem ke kyvadlu. Napište polohu jako funkci času, spočtěte rychlost a zrychlení a načrtněte graf velikosti rychlosti jako funkce času.

\section{Epilog (12 bodů)}
Víťa po návratu hledá krabice s experimenty, které musel schovat v jezeře hlubokém jeden metr. Krabice o rozměrech $30~\mathrm{cm}\times 30~\mathrm{cm}\times 10~\mathrm{cm}$ jsou homogenní s hustotou $1350~\mathrm{kg}\cdot\mathrm{m}^{-3}$. Kolik práce musí vykonat, aby krabici ležící na dně jezera vytáhl těsně nad hladinu?

%\section{Slapské sudy (10 bodů)}
%Nakonec si Víťa nechal nejmenovaného profesora z ÚTF. Vzpomněl si na slavné Orlické vrahy, na profesora si doŠláp a potápí jej do Slap. Zapomněl ovšem válcový sud od Kofoly s profesorem zatížit, ten teď plave na boku a ponořená je pouze jeho polovina. Jakou musí vykonat práci, aby jej plně ponořil a každá část sudu byla alespoň metr pod hladinou? Sud má průměr podstavy 80 cm a výšku 1 metr.

%\section{Výstřel {\huge s}vobody (24 body)}
%Ing. Mgr. Jiří Svoboda, PCH si koupil krém a důkladně promazal svou zbraň. Vypálil svou dávku $\nabla$ženou paní Blaženu.
%
%\section{Investhorium (26 bodů)}
%Ing. Mgr. Jiří Svoboda, PCH si koupil thorium a svítí.
%
%\section{}
%Ing. Mgr. Jiří Svoboda, PCH je v rakvi ve svém krému a thoriu.

%\section*{(Ne)užitečné konstanty}
%\begin{subequations}
%\begin{align*}
%G &= 6.67\cdot 10^{-11}\, \frac{\mathrm{m}^3}{\mathrm{kg}\cdot \mathrm{s}^2} \\
%\mathrm{PCH} &= \text{první certifikovaný hrobník} \\
%\text{poločas rozpadu thoria} &\doteq 14.05\cdot 10^{9}\,\mathrm{yr}
%\end{align*}
%\end{subequations}

\end{multicols}

\end{document}