\documentclass[a4paper, 12pt]{article}
\usepackage[utf8]{inputenc}
\usepackage[czech]{babel}
\usepackage[total={17cm,25cm}]{geometry}
\usepackage{amsmath}
\usepackage{enumitem}

\title{2022 prubezna olympiada mladsi 03}
\author{dolaktomas }
\date{August 2022}

\begin{document}
\begin{center}
    \textbf{\Large Průběžná olympiáda z fyziky mladších}
    \vspace{1em}

    \textbf{termín 22:59:59, 10. 8. 2022 gregoriánského kalendáře}
\end{center}

% \begin{enumerate}[1)]
%     \item nejaky ukol
%     \item jiny ukol
% \end{enumerate}

%\renewcommand{\theenumi}{\arabic{enumi})}
%\begin{enumerate}
\begin{enumerate}[label=\arabic*)]
\setcounter{enumi}{10}
     \item (8 bodů) Navrhněte a předveďte experiment, jehož výsledkem bude pohyb předmětu s nulovým zrychlením a nenulovou rychlostí (která nebude přímo řízena právě vámi!). Připravte si obhajobu (ideálně i výpočet), kterou ukážete, že jste opravdu splnili zadání.
     
     \item (8 bodů) Po ručičkách klasických hodin leze mravenec konstantní rychlostí vůči ručičkám. Mravenec začal lézt o půlnoci po hodinové ručičce ze středu na její konec, který je 12 cm vzdálený od středu, kam se dostal v poledne. V tento čas se přesunul na minutovou ručičku a po ní pokračoval dalších dvanáct hodin na její konec, který je vzdálen 24 cm od středu. Nakreslete grafy závislostí polohy mravence vůči hodinám na čase. Zjistěte okamžité rychlosti mravence v každém bodě časového intervalu.

     \item (6 bodů) Vysvětlete, proč u Newtonovy houpačky neodletí jedna koule rychlostí $2v$ poté, co pustíme dvě koule rychlostí $v$. Všechny koule jsou stejné.

     \item (6 bodů) Vojta si nadhodil volejbalový míč kolmo k povrchu zemskému rychlostí $30\, \mathrm{m}\cdot\mathrm{s}^{-1}$. Podíval se vzhůru, oslnilo ho slunce a padl úžasem k zemi. Jak vysoko proboha ten míč letěl?
     
%     \item (8 bodů) Mějme kulku o hmotnosti 1,5 g, která vyletí z pušky do závaží (Zavrtá se do něj, zdeformuje se a pak pokračuje se závažím v cestě) o hmotnosti 1 kg, které je zavěšené na dlouhém, tenkém lanku zanedbatelné hmotnosti. Závaží se vychýlením zvedlo o 10 cm oproti původní poloze. Jaká byla rychlost střely před srážkou?
     
     \item (2 body) Napište básničku s fyzikální tématikou a předneste ji přede všemi u večeře.
     
     \item (x bodů) Nalezněte chyby ve skriptech a nahlašte to. Bližší podmínky u organizátorů.
     
     \item (3 body) Zdrcený Tomáš prohodil svůj telefon oknem. Popište co nejvíce fyzikálních efektů, které se na dopadu mobilu podílí. Následně zdůvodněte, které lze rozumně zanedbat a proč.

     \item (8 bodů) Spočtěte následující integrály:
     \begin{subequations}
     \begin{align*}
         & \int \frac{x^3 + x^2}{7x^5+x^4-x^7}\,\mathrm{d}x \,,\\
         & \int \frac{1}{\sin^4 x}\,\mathrm{d}x \, .
     \end{align*}
     \end{subequations}
    %  \begin{itemize}
    %      \item $\int \frac{x^3 + x^2}{7x^5+x^4-x^7}$ ,
    %      \item $\int \frac{1}{\sin^4 x}$ .
    %  \end{itemize}
\end{enumerate}

\end{document}
