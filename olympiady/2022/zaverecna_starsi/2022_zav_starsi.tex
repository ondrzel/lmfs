\documentclass[10pt,a4paper,landscape]{article}
\usepackage[utf8]{inputenc}
\usepackage[czech]{babel}
\usepackage[T1]{fontenc}
\usepackage{amsmath}
\usepackage{amsfonts}
\usepackage{amssymb}
\usepackage{graphicx}
\usepackage{multicol}
\usepackage[left=1.5cm,right=1.5cm,top=1.5cm,bottom=1.5cm]{geometry}
\usepackage{enumerate}

\author{Ondřej Zelenka}
\begin{document}
\pagestyle{empty}
\textbf{\center\LARGE Závěrečná paralympiáda starších, LMFS 2022}

\begin{multicols}{2}[]

\section{Lítám v díře}
Mějme v nekonečně hluboké potenciálové jámě na intervalu $x\in\left(0, L\right)$ částici ve stavu
\begin{equation}
\psi\left(x, t=0\right) = N\sin\left(\frac{\pi x}{L}\right)\left[1 + 2\cos\left(\frac{\pi x}{L}\right)\right] \,.
\end{equation}
Vlnovou funkci normujte (určete $N$), rozložte na stacionární stavy a načrtněte hustotu pravděpodobnosti nalezení částice v bodě $x$ v časech $t = 0$ a $t = 2mL^2/\pi\hbar$.

\section{V jiném stavu?}
Mějme několik následovně zadaných vlnových funkcí jednorozměrného kvantově mechanického systému
\begin{equation}
\psi_1\left(x\right) = \frac{1}{\sqrt{\pi}}e^{-x^2} \,,\quad \psi_2\left(x\right) = \frac{1}{\sqrt{\pi}}e^{-x^2+ix} \,,\quad \psi_3\left(x\right) = -\frac{1}{\sqrt{\pi}}e^{-x^2} \,.
\end{equation}
Všechny tyto stavy mají stejnou hustotu pravděpodobnosti $\rho\left(x\right) = \left|\psi\left(x\right)\right|^2$. Znamená to, že se z fyzikálního hlediska jedná o stejné stavy i co se týká např. měření hybnosti nebo časového vývoje? Rozhodněte, které z těchto funkcí popisují stejné a které odlišné stavy.

\section{Zajímavý moment hybnosti}
Mějme nepohybující se částici o spinu 1. Její moment hybnosti je popsán operátory danými maticemi
\begin{equation}
\hat{L}_x = \frac{\hbar}{\sqrt{2}}\begin{pmatrix}0 & 1 & 0\\ 1 & 0 & 1\\ 0 & 1 & 0\end{pmatrix} \,,\quad
\hat{L}_y = \frac{\hbar}{\sqrt{2}}\begin{pmatrix}0 &-i & 0\\ i & 0 &-i\\ 0 & i & 0\end{pmatrix} \,,\quad
\hat{L}_z =       \hbar           \begin{pmatrix}1 & 0 & 0\\ 0 & 0 & 0\\ 0 & 0 &-1\end{pmatrix} \,.
\end{equation}
Částice je ve stavu $\psi = \left(-1,\, i\sqrt{2},\, 1\right)^T$.
\begin{enumerate}[a)]
\item Spočtěte celkový moment hybnosti $\hat{L}^2 = \hat{L}_x^2 + \hat{L}_y^2 + \hat{L}_z^2$ a rozhodněte, zda komutuje s jednotlivými složkami $\hat{L}_x,\,\hat{L}_y,\,\hat{L}_z$.
\item Určete střední hodnoty měření všech tří složek momentu hybnosti. Jedná se o vlastní stav některého z nich?
\end{enumerate}

\section{Koule vole}
Mějme pevnou a nepohyblivou kouli o poloměru 12 metrů na povrchu Země. Stojí na ní skateboardista o zanedbatelných rozměrech a po nekonečně malém šťouchnutí se díky gravitaci rozjíždí dolů. V jaké výšce se od koule odlepí?

\section{Optoelektronový mikroskop}
V jakém intervalu energií je vlnová délka elektronu, popř. neutronu, menší než vlnová délka viditelného světla (od 400 nm)?

\section{Elektron v atomu}
Měříme polohu elektronu v základním stavu v atomu vodíku. Jaké jsou pravděpodobnosti, že jej nalezneme uvnitř sféry o poloměru $a$, a že jej nalezneme mezi sférami o poloměrech $a$ a $2a$, kde $a = 4\pi\hbar^2\epsilon_0/m_eQ_e^2$ je Bohrův poloměr?

\section{Norma}
Normujte stav daný vlnovou funkcí
\begin{equation}
\psi\left(x\right) = \left(2\frac{x^2}{x_0^2} - 1\right)e^{-\frac{x^2}{2x_0^2}} \,,\quad\text{pomůcka:}~\int_{-\infty}^\infty x^{2n}e^{-x^2}\mathrm{d}x = \frac{\left(2n\right)!\sqrt{\pi}}{2^{2n}n!}\,,~ n\in\mathbb{N}
\end{equation}

\section{Záchranná}
Jaký je fyzikální rozměr (tedy jednotka) vlnové funkce elektronu v $d$ dimenzích?

%\section{Odlet (12 bodů)}
%Víťa byl personálem ÚTF deportován z ČR, protože kázal, že $\pi=3$, a $\pi^2=10$. Nejmenovaný profesor napne Víťu o hmotnosti 92 kg na praku popsaném silou $F = -kx$ s tuhostí $k = ???$ na délku $l$ a vystřelí jej po úhlem $45^\circ$. Chce Víťu dostřelit na saskou stranu Krušnohoří 90 kilometrů daleko, o 500 metrů výše. Jaká musí být délka $l$, aby se mu to podařilo?
%
%\section{Sjezd (10 bodů)}
%Víťa si načepoval pivo do válcového kelímku a na skateboardu, který měl v krabici s experimenty, sjíždí z Krušných hor. Silnici tvoří nakloněná rovina pod úhlem $\alpha$, ložiska skateboardu jsou dokonale promazaná a Víťa se bez odporu vzduchu rozjíždí z kopce. Jaký tvar má hladina piva?
%
%\section{Dojezd (8 bodů)}
%Po sjezdu z kopce se silnice srovnala, ale přišla ostrá neklopená zatáčka! Koeficient smykového tření mezi kolečky skateboardu a povrchem silnice je $f=0.1$ a poloměr zatáčky je 10 metrů. Jak rychle může Víťa jet, aby zatáčku vybral?
%
%\section{Kamion (10 bodů)}
%Víťa si v Chemnitz stopnul kamion a cestuje do nizozemského přístavu. V návěsu zavěsil kyvadlo a během rozjíždění s konstantním zrychlením $a$ jej umístil do jeho rovnovážné polohy. Jaká je tato poloha a co se s kyvadlem stane, až přestane pan řidič zrychlovat?
%
%\section{Delfino (8 bodů)}
%Ostatní fámulové se s Víťou setkali v přístavu a pokračují na moře společně. Na lodi uprostřed Atlantiku dostali hlad a rozhodli se ulovit si delfína. Drží síť ve výšce 7 metrů nad hladinou vody a delfín vyskočil z vody přímo do sítě! Jakou musel plavat nejmenší rychlostí?
%
%\section{Ekvádor (10 bodů)}
%Fámulové dorazili do exilu v Ekvádoru a na rovníku sledují show pro turisty, jak se voda v záchodě točí na různou stranu na různých polokoulích. Jako konkurenci sestaví pořádný experiment, ve kterém je tento jev skutečně vidět jako důsledek Coriolisovy síly. Na jakou stranu se bude vír stáčet na které polokouli, a proč?
%%Na jakou stranu na které, a proč? Rozmyslete si důkladně směry a uvažujte, že je Coriolisova síla dostatečně silná, že způsobí znatelný efekt.
%
%\section{Buldoček (10 bodů)}
%Ondra dorazil do Ekvádoru, sleduje falešné předvádění Coriolisovy síly a na pomoc má buldočka - ale jeho pískací otvor se ucpal! Na proražení kruhového otvoru o průměru 1 milimetru je třeba síly ???, jak hluboko je třeba promáčknout stěny, aby se to podařilo? Buldoček má kulový tvar o průměru 10 cm a při zmáčknutí dvěma prsty naproti sobě mají promáčklé strany opět tvar sférických úsečí. Vzduch uvnitř je ideální plyn při tlaku 1 bar.
%
%\section{Sladká pomsta s.r.o. (14 bodů)}
%Víťa se rozhodl pomstít ÚTF za své vyhnanství zničením budovy MFF v Tróji. Hacknul systémy NASA a změnil dráhu ISS, která dosud obíhala 350 km nad zemským povrchem. Jedním krátkým zážehem motorů přímo proti směru pohybu ji zpomalil bez okamžité změny směru rychlosti a její nové přízemí je v přízemí trojské budovy. O kolik změnil Víťa její rychlost?
%
%%\section{Slapské sudy (10 bodů)}
%%Nakonec si Víťa nechal nejmenovaného profesora z ÚTF. Vzpomněl si na slavné Orlické vrahy, na profesora si doŠláp a potápí jej do Slap. Zapomněl ovšem válcový sud od Kofoly s profesorem zatížit, ten teď plave na boku a ponořená je pouze jeho polovina. Jakou musí vykonat práci, aby jej plně ponořil a každá část sudu byla alespoň metr pod hladinou? Sud má průměr podstavy 80 cm a výšku 1 metr.
%
%\section{Výstřel {\huge s}vobody (24 body)}
%Ing. Mgr. Jiří Svoboda, PCH si koupil krém a důkladně promazal svou zbraň. Vypálil svou dávku $\nabla$ženou paní Blaženu.
%
%\section{Investhorium (26 bodů)}
%Ing. Mgr. Jiří Svoboda, PCH si koupil thorium a svítí.
%
%\section{}
%Ing. Mgr. Jiří Svoboda, PCH je v rakvi ve svém krému a thoriu.

\section*{(Ne)užitečné konstanty}
\begin{subequations}
\begin{align*}
G &= 6.67\cdot 10^{-11}\, \frac{\mathrm{m}^3}{\mathrm{kg}\cdot \mathrm{s}^2} \\
\mathrm{PCH} &= \text{první certifikovaný hrobník} \\
\text{poločas rozpadu thoria} &\doteq 14.05\cdot 10^{9}\,\mathrm{yr}
\end{align*}
\end{subequations}

\end{multicols}

\end{document}