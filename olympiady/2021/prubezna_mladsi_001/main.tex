\documentclass[a4paper, 12pt]{article}
\pagestyle{empty}
\usepackage[utf8]{inputenc}
\usepackage[czech]{babel}
\usepackage{amsmath}
\usepackage{amssymb}
\usepackage{enumerate}
\usepackage{mathtools}
\usepackage{graphicx}
\usepackage{amsthm}            %%% veta,definice
\usepackage{multicol}
\usepackage{gensymb}

\newcommand{\dt}{\, \mathrm{d}t}
\newcommand{\de}{\, \mathrm{d}}
\newcommand{\dx}{\, \mathrm{d}x}
\newcommand{\dV}{\, \mathrm{d}V}
\newcommand{\R}{\mathbb{R}}
\newcommand{\E}{\mathbb{E}}
\newcommand{\arctg}{\operatorname{arctg}}
\newcommand{\tg}{\operatorname{tg}}
\newcommand{\cotg}{\operatorname{cotg}}
\newcommand{\grad}{grad}
\newcommand{\dive}{div}
\newcommand{\rot}{rot}

\usepackage{mathtools}

\newcommand\myeq{\stackrel{\mathclap{\normalfont\mbox{def}}}{=}}


\begin{document}


\begin{center}
\textbf{{\large
Průběžná fyzikální olympiáda -- mladší\\
Deadline -- 20. 8. 2021 23:00
}}

\end{center}


\begin{enumerate} [1)]
\item\relax[5 bodů] Mějme vektory $\Vec{u}= (4, 5.5, \frac{2}{3})$ a $\Vec{v}=(1,0,3)$. Zjistěte, jaký úhel spolu svírají. Určete vektory $\Vec{w}=\Vec{u} \times \Vec{v}$ a $\Vec{a} = \Vec{v} \times \Vec{u}$. Vytvořte vektor $\Vec{b}$, tak aby vektory $\Vec{v}, \Vec{w}$ a $\Vec{b}$ tvořily pravotočivou ortogonální bázi (soustavu navzájem kolmých vektorů). Nalezněte konstanty $V, W$ a $B$, takové, aby platilo $\|V\Vec{v}\|=1, \|W\Vec{w}\|=1$ a $\|B\Vec{b}\|=1$.

\item\relax[3 body] Mějme body $A=[2,1]$ a $B=[3,y]$. Napište jejich souřadnice v polární soustavě souřadnic. Otočte oba body o $5^{\circ}$ a napište jejich kartézské souřadnice.

\item\relax[3 body] Zderivujte následující fce:
    \begin{enumerate}[1)]
         \item $$\frac{\de}{\dx} \frac{1}{\log (x)}$$
         \item $$\frac{\de}{\dx} e^{x^x} $$
         \item $$\frac{\de}{\dx} \cos(\sin(\log e^\pi) $$
    \end{enumerate}

\end{enumerate}

\end{document}
