\documentclass[a4paper, 12pt]{article}
\pagestyle{empty}
\usepackage[utf8]{inputenc}
\usepackage[czech]{babel}
\usepackage{amsmath}
\usepackage{amssymb}
\usepackage{enumerate}
\usepackage{mathtools}
\usepackage{graphicx}
\usepackage{amsthm}            %%% veta,definice
\usepackage{multicol}
\usepackage{gensymb}

\newcommand{\dt}{\, \mathrm{d}t}
\newcommand{\dx}{\, \mathrm{d}x}
\newcommand{\dV}{\, \mathrm{d}V}
\newcommand{\R}{\mathbb{R}}
\newcommand{\E}{\mathbb{E}}
\newcommand{\arctg}{\operatorname{arctg}}
\newcommand{\tg}{\operatorname{tg}}
\newcommand{\cotg}{\operatorname{cotg}}
\newcommand{\grad}{grad}
\newcommand{\dive}{div}
\newcommand{\rot}{rot}

\usepackage{mathtools}

\newcommand\myeq{\stackrel{\mathclap{\normalfont\mbox{def}}}{=}}


\begin{document}


\begin{center}
\textbf{{\large
Průběžná fyzikální olympiáda -- starší\\
Deadline -- 16. 8. 2021 23:00
}}

\end{center}


\begin{enumerate} [1)]
\item\relax[1 bod] Napište předpis pro zrychlení rovnoměrného harmonického pohybu po kružnici. 
\item\relax[2 body] Napište předpis pro trajektorii rovnoměrného harmonického pohybu po kružnici, když v čase $t_0=0$ má objekt souřadnice $[1,1]$.
\item\relax[3 body] Napište předpis pro diferenciální operátory $\grad(\varphi)$, $\dive(\Vec{u})$, $\rot(\Vec{u})$ a $\Delta(\varphi)$.
\item\relax[5 bodů] Napište Maxwellovy rovnice v diferenciálním i integrálním tvaru a vlastními slovy popište, co vyjadřují.
\end{enumerate}

\end{document}
