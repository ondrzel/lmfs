\documentclass[10pt,a4paper,landscape,twocolumn]{article}
\usepackage[utf8]{inputenc}
\usepackage[czech]{babel}
\usepackage[T1]{fontenc}
\usepackage{amsmath}
\usepackage{amsfonts}
\usepackage{amssymb}
\usepackage{graphicx}
\usepackage[left=1.5cm,right=1.5cm,top=1.5cm,bottom=1.5cm]{geometry}
\author{Ondřej Zelenka}
\title{Závěrečná paralympiáda mladších 2021}

% nastavit číslování úloh bez 0 jakožto čísla sekce na začátku
\renewcommand{\thesubsection}{\arabic{subsection}.}

% vytvořit counter na body a definovat příkazy pro skloňování slova "bod" a tvorbu úlohy
\newcounter{bodycounter}
\newcommand{\bodystring}[1]{\ifnum #1=1 1 bod\else\ifnum #1<5 #1 body\else #1 bodů\fi\fi}
\newcommand{\uloha}[3]{
\subsection{#1 (\bodystring{#2})}
#3\addtocounter{bodycounter}{#2}}

\pagestyle{empty}
\begin{document}
\section*{Závěrečná paralympiáda mladších LMFS 2021}

\uloha{Kvalitní guma}{10}{
Filip testuje pevnost prcgumy. Dle informací od výrobce se síla řídí vzorcem
\begin{equation}
F = -kx^3 ~,
\end{equation}
kde $x$ je vzdálenost, o kterou je šprcka protažená, a $k$ je konstanta. Na hřišti připevnil jeden konec ke kládě a s druhým koncem v ruce vyběhl rychlostí ???. Setrvačností doběhl do vzdálenosti 8 metrů, kde ho kupředu již nic netlačilo a zpět ho táhla síla prezervativu. Jakou má tuhost?
}

%\uloha{Nenažranec}{10}{
%Ondra běží na Královku. Po rovině těsně před začátkem kopce v nadmořské výšce $731$ m běží tempem 4:30 na kilometr a na vrchol $858$ m.n.m sotva došlapuje tempem 12:00 na kilometr. Kolik tyčinek M{\"u}sli na zdraví s borůvkami a malinami o energetické hodnotě 1607 kJ musí Ondra sníst, aby doplnil energii?
%}

\uloha{Méďa Béďa}{10}{
Ondra zvaný Bedřich se rozhodl zdolat rekord světa ve skoku vysokém. Jedná se o zdatného fyzika, tudíž se rozhodl pro svůj úspěch využít svých znalostí, rekord provede na Měsíci. Dokáže dosáhnout rychlosti $5~\mathrm{m\cdot s^{-1}}$. Jak vysoko dokáže dostat své těžiště? Ve stoje má těžiště ve výšce 1 m nad povrchem kde stojí.
}

\uloha{Tripl kripl}{10}{
S jakou přesností je třeba určit úhel pro hod šipkami, chceme-li trefit triple dvacet?
}

\uloha{$\kappa$}{10}{
Jak se změní potenciál(ní energie) nejmenovaného účastníka, když k němu přijde nejmenovaná účastnice a oni se z pozice vzpřímené přemístí do pozice vleže? Těžiště v poloze ve stoje má v 80 cm, v poloze vleže 60 cm a jeho hmotnost je 80 kg.
}

\uloha{Jasný bod na obloze}{10}{
Mravenec leze po minutové ručičce hodin rovnoměrně přímočaře rychlostí 0.005 m/s. Sestrojte graf závislosti polohy na čase, graf závislosti rychlosti na čase a graf závislosti zrychlení na čase mravence z pohledu soustavy spojené se zemí.
}

\uloha{Baterka}{10}{
Baterie má kapacitu (tj. množství uložené energie) 60 Wh. Vypočítejte, do jaké výšky dokáže elektromotor připojený k této baterii vyzdvihnout závaží o hmotnosti 60 kg za předpokladu, že má účinnost 60\%.
}

\uloha{Střela střelcem}{10}{
Víťa s Ondrou hráli šachy. Víťa vyhrál a Ondra se naštval. Hodil po Víťovi střelce, ale netrefil se a střelec prolétnul kolem Víťovy hlavy. Spočítejte, jakou rychlostí Ondra hodil střelce, když ho házel z výšky 2 m vodorovným směrem a dopadl do vzdálenosti 20 m. Tření vzduchu zanedbejte.
}

\uloha{Harmonický život}{10}{
Pan Ing. Svoboda se rozhodl svůj hlas položit co nejníž a zaplatil si bungee jumping z mostu. Jak tuhá musí být pružina, aby nebyl tuhý pan inženýr a osciloval s periodou pět sekund?
}

\uloha{Sihnus a cosihnus}{10}{
Spočtěte následující integrály:
\begin{subequations}
\begin{align}
&\int_0^{\pi/2} \frac{\sin\left(x\right)}{\sqrt[17]{\cos\left(x\right)}}\mathrm{d} x ~,\\
&\int_0^{2\pi} \sin\left(x\right)\cdot\sin\left(\cos\left(\tanh\left(\Gamma\left(\pi^{14}\right)\right)\right)\right)\mathrm{d} x ~,
\end{align}
\end{subequations}
kde $\Gamma\left(z\right) = \int_0^\infty x^{z-1}e^{-x}\mathrm{d}x$ je gama funkce.
}

%\uloha{Kosmické smetí}{10}{
%Kačák se bude bourat. Demoliční firma však nemá kam odvézt sutiny, tak se rozhodla vyhodit Kačák do povětří tak, aby obíhal po orbitě kolem Země.
%}

\subsection*{Užitečné konstanty}
Indexy lomu: vakuum 1, vzduch 1.00026, voda 1.33 \\
Vlnové délky ve vzduchu: zelená 532 nm, tmavě oranžová 600 nm\\
Hmotnost: Ondra 65 kg, Ing. Jiří Svoboda 90 kg, Filip 86 kg

\newpage Celkem \arabic{bodycounter} bodů.

\end{document}