\documentclass[10pt,a4paper,landscape,twocolumn]{article}
\usepackage[utf8]{inputenc}
\usepackage[czech]{babel}
\usepackage[T1]{fontenc}
\usepackage{amsmath}
\usepackage{amsfonts}
\usepackage{amssymb}
\usepackage{graphicx}
\usepackage[left=1.5cm,right=1.5cm,top=1.5cm,bottom=1.5cm]{geometry}
\author{Ondřej Zelenka}
\title{Závěrečná paralympiáda mladších 2021}

% nastavit číslování úloh bez 0 jakožto čísla sekce na začátku
\renewcommand{\thesubsection}{\arabic{subsection}.}

% vytvořit counter na body a definovat příkazy pro skloňování slova "bod" a tvorbu úlohy
\newcounter{bodycounter}
\newcommand{\bodystring}[1]{\ifnum #1=1 1 bod\else\ifnum #1<5 #1 body\else #1 bodů\fi\fi}
\newcommand{\uloha}[3]{
\subsection{#1 (\bodystring{#2})}
#3\addtocounter{bodycounter}{#2}}

\pagestyle{empty}
\begin{document}
\section*{Závěrečná paralympiáda mladších LMFS 2021}

\uloha{Kvalitní guma}{10}{
Síla je
\begin{equation}
F = -kx^3
\end{equation}
}

\uloha{Jasný bod na obloze}{10}{
Ondra běží na Královku. Po rovině těsně před začátkem kopce v nadmořské výšce $731$ m běží rychlostí $14~\mathrm{km\cdot h^{-1}}$ a na vrchol $858$ m.n.m sotva došlapuje rychlostí $5~\mathrm{km\cdot h^{-1}}$. Kolik tyčinek M{\"u}sli na zdraví s borůvkami a malinami o energetické hodnotě 1607 kJ musí Ondra sníst, aby doplnil energii?
}

\uloha{Jasný bod na obloze}{10}{
Tady je text úlohy!
}

\uloha{Ponorky Revolutions}{10}{
Tady je text úlohy!
}

\uloha{Jasný bod na obloze}{10}{
Tady je text úlohy!
}

\uloha{Jasný bod na obloze}{10}{
Tady je text úlohy!
}

\uloha{Jasný bod na obloze}{10}{
Tady je text úlohy!
}

\uloha{Jasný bod na obloze}{10}{
Tady je text úlohy!
}

\uloha{Jasný bod na obloze}{10}{
Tady je text úlohy!
}

\uloha{Jasný bod na obloze}{10}{
Tady je text úlohy!
}

\subsection*{Užitečné konstanty}
Indexy lomu: vakuum 1, vzduch 1.00026, voda 1.33 \\
Vlnové délky ve vzduchu: zelená 532 nm, tmavě oranžová 600 nm\\
Ondrova hmotnost: 65 kg

\newpage Celkem \arabic{bodycounter} bodů.

\end{document}