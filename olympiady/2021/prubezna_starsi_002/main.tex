\documentclass[a4paper, 12pt]{article}
\pagestyle{empty}
\usepackage[utf8]{inputenc}
\usepackage[czech]{babel}
\usepackage{amsmath}
\usepackage{amssymb}
\usepackage{enumerate}
\usepackage{mathtools}
\usepackage{graphicx}
\usepackage{amsthm}            %%% veta,definice
\usepackage{multicol}
\usepackage{gensymb}

\newcommand{\dt}{\, \mathrm{d}t}
\newcommand{\dx}{\, \mathrm{d}x}
\newcommand{\dV}{\, \mathrm{d}V}
\newcommand{\R}{\mathbb{R}}
\newcommand{\E}{\mathbb{E}}
\newcommand{\arctg}{\operatorname{arctg}}
\newcommand{\tg}{\operatorname{tg}}
\newcommand{\cotg}{\operatorname{cotg}}
\newcommand{\grad}{\mathrm{grad}~}
\newcommand{\dive}{\mathrm{div}~}
\newcommand{\rot}{\mathrm{rot}~}

\usepackage{mathtools}

\newcommand\myeq{\stackrel{\mathclap{\normalfont\mbox{def}}}{=}}

\begin{document}


\begin{center}
\textbf{{\large
Průběžná fyzikální olympiáda -- starší\\
Deadline -- 24. 8. 2021 23:00
}}

\end{center}


\begin{enumerate} [1)]
\setcounter{enumi}{4}
\item\relax[5 bodů] Vypočtěte s pomocí dvojného integrálu obsah kruhu o poloměru 3 metry. 
\item\relax[6 bodů] Dokažte, že platí:

\begin{enumerate}[a)]
    \item $\rot\rot\vec{A}=\grad\dive\vec{A} - \Delta\vec{A}$
    \item $\rot\grad \varphi = 0$
    \item $\dive\rot\vec{A}= 0$
\end{enumerate}
\item\relax[10 bodů]
Napište obecnou skalární funkci času a prostorových souřadnic, která popisuje válcovou vlnu. Hustotu energie spojenou s touto vlnou považujte za přímo úměrnou druhé mocnině hodnoty hledané funkce v daném místě a čase.
\item\relax[5 bodů] Ukažte, že funkce $\psi(t\mp\frac{\vec{s}\cdot \vec{r}}{v})$, kde $\|\vec{s}\|=1$, představuje rovinnou vlnu postupující v prostoru směrem $\pm\vec{s}$ rychlostí $v$.

\item\relax[2 body] Určete frekvence elektromagnetických vln ve viditelném spektru vlnových délek od 391 do 779 nm.
\item\relax[3 body] Disperzní křivka skla může být přibližně vyjádřena Cauchyovým empirickým vzorcem $n=A + B\cdot \lambda^{-2}$. Najděte hodnoty fázové a grupové rychlosti pro sklo, u nějž jsou $A=1,40$ a $B=2,5\cdot10^{-14}$ m$^2$. 
\end{enumerate}

\end{document}
