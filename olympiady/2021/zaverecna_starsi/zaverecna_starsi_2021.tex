\documentclass[10pt,a4paper,landscape,twocolumn]{article}
\usepackage[utf8]{inputenc}
\usepackage[czech]{babel}
\usepackage[T1]{fontenc}
\usepackage{amsmath}
\usepackage{amsfonts}
\usepackage{amssymb}
\usepackage{graphicx}
\usepackage[left=1.5cm,right=1.5cm,top=1.5cm,bottom=1.5cm]{geometry}
\author{Ondřej Zelenka}
\title{Závěrečná paralympiáda mladších 2021}

% nastavit číslování úloh bez 0 jakožto čísla sekce na začátku
\renewcommand{\thesubsection}{\arabic{subsection}.}

% vytvořit counter na body a definovat příkazy pro skloňování slova "bod" a tvorbu úlohy
\newcounter{bodycounter}
\newcommand{\bodystring}[1]{\ifnum #1=1 1 bod\else\ifnum #1<5 #1 body\else #1 bodů\fi\fi}
\newcommand{\uloha}[3]{
\subsection{#1 (\bodystring{#2})}
#3\addtocounter{bodycounter}{#2}}

\pagestyle{empty}
\begin{document}
\section*{Závěrečná paralympiáda starších LMFS 2021}

\uloha{Antireflexní brýle}{12}{
Tomáš má brýle ze skla s indexem lomu 1.52, ale odrážejí podle něj příliš mnoho světla. Aplikoval na ně tedy vrstvu $\mathrm{MgF_2}$ o indexu lomu 1.38, aby mezi světlem odraženým rozhraním vzduch-$\mathrm{MgF_2}$ a světlem odraženým rozhraním $\mathrm{MgF_2}$-sklo došlo k destruktivní interferenci. Jak silnou vrstvu má zvolit, aby úplně odstranil odrazy kolmo dopadajícího zeleného světla?
}

\uloha{Ponorky}{10}{
Při honu za vlajkou se jedna z ponorek poněkud vymkla kontrole, ponořila se do nádrže Josefův důl a dále hledala vlajku. Když to konečně na dně v hloubce 10 metrů vzdala, začala svítit baterkou o záchranu. Jak blízko musí připlout Pobřežní hlídka, aby ji našla? Uvažujte, že David Hasselhoff má oči přímo nad dokonale klidnou hladinou a spočítejte vzdálenost od bodu na hladině přímo nad ponorkou.
}

\uloha{Ponorky Reloaded}{8}{
Při záchraně ponorky ve vodě ztratila Pamela Anderson své tmavě oranžové plavky. Požádala Víťu o pomoc, a oba hledají oranžové plavky na dně. Víťa ovšem zapomněl, že světlo při přechodu do prostředí s jiným indexem lomu mění vlnovou délku a nezamyslel se, kterou barvu tedy má hledat. Pokud i ve vodě hledá stejnou tmavě oranžovou, najde je? Důkladně zdůvodněte!
}

\uloha{Ponorky Revolutions}{10}{
Tady je text úlohy!
}

\uloha{Jasný bod na obloze}{10}{
Ondra se dívá z přístavu v Bregenz přes Bodamské jezero a vyhlíží městské zahrady v Kostnici, které jsou přímou čarou 45.81 km daleko. Jak musí být nejméně vysoký, aby je měl šanci zahlédnout alespoň s velmi výkonným dalekohledem? Uvažujte, že Ondra stojí 5 metrů nad hladinou jezera, a stejně vysoko jsou i zahrady.
}

\uloha{Jasný bod na obloze}{10}{
Tady je text úlohy!
}

\uloha{Jasný bod na obloze}{10}{
Tady je text úlohy!
}

\uloha{Jasný bod na obloze}{10}{
Tady je text úlohy!
}

\uloha{Jasný bod na obloze}{10}{
Tady je text úlohy!
}

\uloha{Jasný bod na obloze}{10}{
Tady je text úlohy!
}

\subsection*{Užitečné konstanty}
Indexy lomu: vakuum 1, vzduch 1.00026, voda 1.33 \\
Vlnové délky ve vzduchu: zelená 532 nm, tmavě oranžová 600 nm

\newpage Celkem \arabic{bodycounter} bodů.

\end{document}