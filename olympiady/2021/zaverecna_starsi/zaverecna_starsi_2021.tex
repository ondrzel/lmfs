\documentclass[11pt,a4paper,landscape,twocolumn]{article}
\usepackage[utf8]{inputenc}
\usepackage[czech]{babel}
\usepackage[T1]{fontenc}
\usepackage{amsmath}
\usepackage{amsfonts}
\usepackage{amssymb}
\usepackage{graphicx}
\usepackage[left=1.5cm,right=1.5cm,top=1.5cm,bottom=1.5cm]{geometry}
\author{Ondřej Zelenka}
\title{Závěrečná paralympiáda mladších 2021}

\usepackage{color}

% nastavit číslování úloh bez 0 jakožto čísla sekce na začátku
\renewcommand{\thesubsection}{\arabic{subsection}.}

% vytvořit counter na body a definovat příkazy pro skloňování slova "bod" a tvorbu úlohy
\newcounter{bodycounter}
\newcommand{\bodystring}[1]{\ifnum #1=1 1 bod\else\ifnum #1<5 #1 body\else #1 bodů\fi\fi}
\newcommand{\uloha}[3]{
\subsection{#1 (\bodystring{#2})}
#3\addtocounter{bodycounter}{#2}}

\pagestyle{empty}
\begin{document}
\section*{Závěrečná paralympiáda starších LMFS 2021}

\uloha{Antireflexní brýle}{14}{
Tomáš má brýle ze skla s indexem lomu 1.52, ale odrážejí podle něj příliš mnoho světla. Aplikoval na ně tedy vrstvu $\mathrm{MgF_2}$ o indexu lomu 1.38, aby mezi světlem odraženým rozhraním vzduch-$\mathrm{MgF_2}$ a světlem odraženým rozhraním $\mathrm{MgF_2}$-sklo došlo k destruktivní interferenci. Jak silnou vrstvu má zvolit, aby úplně odstranil odrazy kolmo dopadajícího zeleného světla?
}

\uloha{Ponorky}{8}{
Při honu za vlajkou se jedna z ponorek poněkud vymkla kontrole, ponořila se do nádrže Josefův důl a dále hledala vlajku. Když to konečně na dně v hloubce 10 metrů vzdala, začala svítit baterkou o záchranu. Jak blízko musí připlout Pobřežní hlídka, aby ji našla? Uvažujte, že David Hasselhoff má oči přímo nad dokonale klidnou hladinou a spočítejte vzdálenost od bodu na hladině přímo nad ponorkou.
}

\uloha{Ponorky Reloaded}{8}{
Při záchraně ponorky ve vodě ztratila Pamela Anderson své tmavě oranžové plavky. Požádala Víťu o pomoc, a oba hledají oranžové plavky na dně. Víťa ovšem zapomněl, že světlo při přechodu do prostředí s jiným indexem lomu mění vlnovou délku a nezamyslel se, kterou barvu tedy má hledat, a vyhlíží stejnou tmavě oranžovou. Když se mu to nepodařilo, bylo to protože hledal špatnou barvu, nebo protože se příliš soustředil na něco jiného? Důkladně zdůvodněte!
}

\uloha{Ponorky Revolutions}{10}{
Plavky sežral josefodolský delfín a plave přehradou rychlostí $v = ???$. Nad přehradou je 20 obručí rozmístěných lineárně po 100 km do výšky 2000 km. Kolik z nich dokáže delfín proskočit? \textcolor{green}{Dopočítat rychlost, aby to vycházelo 14.73.}
}

\uloha{Chrastí posměšně zápalkami}{7}{
Ondra se dívá z přístavu v Bregenz přes Bodamské jezero a vyhlíží městské zahrady v Kostnici, které jsou 45.81 km daleko. Jak musí být nejméně vysoký, aby je měl šanci zahlédnout alespoň s velmi výkonným dalekohledem? Uvažujte, že Ondra stojí 5 metrů nad hladinou jezera, a stejně vysoko jsou i zahrady.
}

\uloha{Jasný bod na stínítku}{8}{
Víťa si z AliExpressu objednal laser neznámé barvy a potřebuje zjistit jeho vlnovou délku. Sestavil tedy Youngův experiment se štěrbinami $d = 0.5$ cm od sebe a stínítkem $a = 5$ m za štěrbinami. Změřil, že první interferenční maximum je od středního pruhu vzdáleno $p = 0.05$ cm. Určete vlnovou délku světla.
}

\uloha{V Tomášově stínu}{10}{
Tomáš stojí v Bedřichovské přehradě tak, že celé jeho nohy o délce 90 cm jsou v ní ponořené. Jak dlouhý stín vrhají jeho nohy na dno přehrady, jestliže sluneční paprsky dopadají na vodní hladinu pod úhlem $60^\circ$ od kolmice? Změní se délka stínu jeho nohou, když vyleze na souš, a jak?
}

\uloha{$\pi$érh the Ferhma}{12}{
Světelný paprsek na cestě z bodu A do bodu B proniká rozhraním dvou prostředí o různých indexech lomu. Je natolik chytrý, že se na rozhraní zlomí tak, aby cestu urazil za co nejkratší čas. Vyjádřete výsledek pomocí úhlů ke kolmici a srovnejte se Snellovým zákonem.
}

\uloha{Slizoun}{12}{
Tomáš plave přes Bedřichovskou přehradu a vyvíjí stálý výkon $P = ???$. Brzdí ho Stokesův odpor ve tvaru $F_d = 6\pi \mu Rv$, kde $R = 1 m$ je poloměr ideálně kulového Tomáše, $v$ je jeho rychlost a $\mu$ je dynamická viskozita, která se díky postupnému obalování těla slizem během 20 minut lineárně v čase sníží na nulu. Jak daleko za tu dobu doplave? \textcolor{green}{Spočítat $P$ tak, aby vyšlo 385 metrů!}
}

\uloha{Fotony}{8}{
Na zemské oběžné dráze kolem Slunce má sluneční záření hustotu $1361~\mathrm{W\cdot m^{-2}}$. Kolik kilo světla dopadne za rok?
}

\subsection*{Užitečné konstanty}
Indexy lomu: vakuum 1, vzduch 1.00026, voda 1.33 \\
Vlnové délky ve vzduchu: zelená 532 nm, tmavě oranžová 600 nm\\
Viskozita vody: $\mu = 1.0016~\mathrm{mPa\cdot s}$\\
Poloměr Země: 6378 km

\newpage Celkem \arabic{bodycounter} bodů.

\end{document}