\documentclass[a4paper, 12pt]{article}
\pagestyle{empty}
\usepackage[utf8]{inputenc}
\usepackage[czech]{babel}
\usepackage{amsmath}
\usepackage{amssymb}
\usepackage{enumerate}
\usepackage{mathtools}
\usepackage{graphicx}
\usepackage{amsthm}            %%% veta,definice
\usepackage{multicol}
\usepackage{gensymb}

\newcommand{\dt}{\, \mathrm{d}t}
\newcommand{\de}{\, \mathrm{d}}
\newcommand{\dx}{\, \mathrm{d}x}
\newcommand{\dV}{\, \mathrm{d}V}
\newcommand{\R}{\mathbb{R}}
\newcommand{\E}{\mathbb{E}}
\newcommand{\arctg}{\operatorname{arctg}}
\newcommand{\tg}{\operatorname{tg}}
\newcommand{\cotg}{\operatorname{cotg}}
\newcommand{\grad}{grad}
\newcommand{\dive}{div}
\newcommand{\rot}{rot}

\usepackage{mathtools}

\newcommand\myeq{\stackrel{\mathclap{\normalfont\mbox{def}}}{=}}


\begin{document}


\begin{center}
\textbf{{\large
Průběžná fyzikální olympiáda -- mladší\\
Deadline -- 25. 8. 2021 23:00
}}

\end{center}


\begin{enumerate} [1)]
\setcounter{enumi}{3}
\item\relax[5 bodů] 
Z parlamentu padá Volný pádem se zrychlením $\Vec{g}=(0,0,-9.81)$~m$\cdot$s$^{-2}$ (z druhého patra, tj. 7~m nad Zemí). Popište průběh rychlosti a zrychlení uvažovaného Volného pádu. 
\item\relax[3 body] 

Vypočtěte s pomocí integrálu, jaký obsah má elipsa se středem v počátku, když její hlavní poloosa má velikost $a$ a vedlejší poloosa má velikost $b$.

\item\relax[3 body] 

Najděte primitivní funkci (ale ne moc hloupou) k funkcím:
\begin{enumerate}[a)]
    \item $\int x \sin x^2 \de x$
    \item $\int y \sin^2 (y^2) \cos(y^2) \de y$
\end{enumerate}

\end{enumerate}

\end{document}
